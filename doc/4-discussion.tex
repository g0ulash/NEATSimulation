\section{Discussion}

We found a significant difference between the average crossover in a dynamic environment and a static environment.
The results showed that sexual reproduction gets selected more in a dynamic environment than in a static environment.
This means that the findings that the earlier research had (i.e. in \cite{misevicchanging}) also extend to a more ecological valid simulation, using neural networks instead of the Avida simulation.
More extensive research where a more complex, more realistic environment could be interesting to find which influence sexual reproduction has on evolution.

Furthermore, looking at the descriptive results, a few interesting observations can be made.
Figure \ref{fig:avgcross} shows the average crossover of the two different (dynamic and static) conditions over time.
As explained in the results, the average crossover for the static condition declines slowly but steadily over time.
Now two observations can be made here, the first being that it would be interesting to see what happens to the average crossover when the simulation was run for more than the $300,000$ iterations that have been run in this experiment.
If it keeps declining over time, it could show that eventually asexual reproduction is more advantageous in a static environment than sexual reproduction is - this should be tested with a different research question than in this experiment.
One of the reasons that potentially explain the decrease is that once the environment is known by the agents, i.e. they have learned their optimal solution, and there are no changes in the environment, the cost of sexual reproduction is too high to keep sustaining it compared to only asexual reproduction.
A new experiment that investigates the influence of the parameter such as crossover cost could point out in how far the cost of sexual reproduction is too high to sustain that type compared to asexual reproduction.
The second observation that can be made is that the declining of the average crossover perhaps can be influenced by the set parameters of the simulation.
An interesting experiment that could be looked at, would be to compare different dynamic conditions, where the food swaps are performed at different iterations, and see what influence it has on the average crossover rate and if it also shows a tendency to converge for other dynamic conditions.

Looking at Figure \ref{fig:avgfit}, it shows that the average fitness increases consistently in the static condition but stays at low values in the dynamic conditions.
Future research could investigate how to overcome this low fitness problem in the dynamic conditions.
The type of learning that is used in the simulation is something that can increase the average fitness.
In the case of the neural networks and the learning that is used right now, is that actually only the children of sexual reproduction can learn from the mistakes of their parents.
Perhaps combining the current experiment with reinforcement learning could allow the agents to learn from their mistakes more easily and eventually increase the fitness. %I am not sure if I put it correctly like this?

A final remark concerning the algorithm used in this experiment, is that future research could look into algorithms that are extensions of NEAT.
Speciation was also not implemented for the current experiment as it did not seem required, however it could be interesting to look at the effect that speciation has on the fitness.
Another alternative algorithm is HyperNEAT, which shows a few interesting improvements compared to NEAT \cite{stanleyhypercube}.
HyperNEAT is able to evolve neural networks that look more like the neural and structural connectivity patterns as found in the brain, which enhances the validity of an Artificial Life - especially concering the ecological validitiy.
The second important feature that HyperNEAT offers, is that it can see the geometry of the problem domain that it is trying to solve.
It can, compared to other artifical neural networks, learn instantly to exploit the geometry that it is in, which can increase the learning significantly.
Especially the last feature could be interesting since the position of the food and poison are influental on the fitness of the agents.