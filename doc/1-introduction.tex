\section{Introduction}

There are many theories on the evolution of sex and on the advantages of sexual reproduction over asexual reproduction \cite{misevic}. 
It remains an open question as to why sexual reproduction is so common in the natural world despite the relatively higher cost over asexual reproduction. 
The paper by Dusan Misevic et al. \cite{misevicchanging} tests the hypothesis that sexual reproduction is advantageous in changing environments using digital organisms. 
The digital organisms were simulated in Avida, in which the organisms are short self-replicating computer programs. 
They then modified the metabolic values of poisons and nutrients such that they could switch and change in value. 
They found that sex became the dominant mode of reproduction when the environmental change was rapid and substantial enough. 
Thus the study finds support for the importance of changing environments in the evolution of sex. 

The review paper by Matthew hartfield and Peter D. Keightley \cite{matthewhartfield} gives an overview of some of the dominant theories on the evolution of sex. 
The four classes of theories discussed are: The Fisher-Muller hypothesis, Muller's ratchet, Red Queen hypothesis and the mutational deterministic hypothesis.
The Fisher-Muller hypothesis states that sexual reproduction allows natural selection to work faster than when reproducing asexually since recombination can bring together advantageous genes faster than asexual reproduction could.
Muller's ratchet states that sex evolved to fix multiple advantageous mutants as a mechanism to stop the build-up of deleterious mutations in finite populations. 
In asexual reproduction bad genes can build up, known as genetic load, which would lead to an organism going extinct. 
Sexual reproduction could counter this by recombination, stopping the build-up of these bad genes.
The Red Queen hypothesis is the hypothesis that sexual reproduction evolved due to co-evolution of opposing organisms in a rapidly changing environment. 
Sex allows for faster adaptation to these changing environments and gives a higher chance of survival.
Mutational deterministic hypothesis assumes that the majority of deleterious genes are only slightly bad, but that each additional bad gene increasingly reduces the fitness of the organism. 
Sexual reproduction will generate offspring with fewer deleterious genes and some with more, those that have more will quickly die out but those with few will survive. 
Thus sex can help remove these slightly deleterious genes that only combined have a large effect.
They conclude that theoretical and experimental results favor the hypothesis that sex allows for the recombination of advantageous genes (The Fisher-Muller hypothesis), or acts as a mechanism to shuffle genotypes in order to repel parasitic invasion (Red Queen hypothesis) which is in line with our main hypothesis.

One hypothesis, as stated before, is that sexual reproduction does better in a faster changing environment \cite{misevicchanging}. 
One explanation for this is that sexual reproduction allows for faster adaptation in a fast changing environment due to recombination. 
With recombination groups of genes can be swapped faster which would result in higher variations and a higher chance of adapting to the new environment.
The study however is not very ecologically valid, since it performs on logical operations and its environment does not match the real world. 
We therefore will use NEAT \cite{stanleyneat}, an evolutionary algorithm which operates on neural networks, in a simulated environment. 
Although the simulation is a gross simplification it is still closer to the real world than Avida and should give better insight into the advantages of sexual over asexual reproduction.
The proposed research question is as follows:

\begin{quote}
Is sexual reproduction more advantageous in a faster changing Artificial Life environment than asexual reproduction where the agents are neural networks evolved with NEAT? 
\end{quote}

\noindent Based on earlier research and the fact that the proposed environment is an even better suit for simulations, we hypothesize that sexual reproduction is more advantageous in a faster changing environment than asexual reproduction is.

