\section{Introduction}

There are many theories on the evolution of sex and on the advantages of sexual reproduction over asexual reproduction \cite{misevic}. 
It remains an open question as to why sexual reproduction is so common in the natural world despite the relatively higher cost over asexual reproduction. 
One hypothesis, and the hypothesis that will be tested in this project, is that sexual reproduction does better in a faster changing environment \cite{misevicchanging}. 
One explanation for this is that sexual reproduction allows for faster adaptation in a fast changing environment due to recombination. 
With recombination groups of genes can be swapped faster which would result in higher variations and a higher chance of adapting to the new environment. 
This hypothesis has been explored before using Avida \cite{misevicchanging}. 
The study however is not very ecologically valid, since it performs on logical operations and its environment does not match the real world. 
We therefore will use NEAT \cite{stanleyneat}, an evolutionary algorithm which operates on neural networks, in a simulated environment. 
Although the simulation is a gross simplification it is still closer to the real world than Avida and should give better insight into the advantages of sexual over asexual reproduction. 